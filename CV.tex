%%%%%%%%%%%%%%%%%%%%%%%%%%%%%%%%%%%%%%%%%
% Medium Length Professional CV
% LaTeX Template
% Version 2.0 (8/5/13)
%
% This template has been downloaded from:
% http://www.LaTeXTemplates.com
%
% Original author:
% Rishi Shah 
%
% Important note:
% This template requires the resume.cls file to be in the same directory as the
% .tex file. The resume.cls file provides the resume style used for structuring the
% document.
%
%%%%%%%%%%%%%%%%%%%%%%%%%%%%%%%%%%%%%%%%%

%----------------------------------------------------------------------------------------
%	PACKAGES AND OTHER DOCUMENT CONFIGURATIONS
%----------------------------------------------------------------------------------------

\documentclass{resume} % Use the custom resume.cls style

\usepackage[left=0.75in,top=0.6in,right=0.75in,bottom=0.6in]{geometry} % Document margins
\usepackage{hyperref}
\newcommand{\tab}[1]{\hspace{.2667\textwidth}\rlap{#1}}
\newcommand{\itab}[1]{\hspace{0em}\rlap{#1}}
\name{Katherine Laliotis} % Your name
\address{(858) 245-3412 $|$ kklaliotis@gmail.com $|$ ORCiD: 0000-0002-6111-6061
}
 % Your address
%\address{123 Pleasant Lane \\ City, State 12345} % Your secondary addess (optional)
 % Your phone number and email

\begin{document}

%----------------------------------------------------------------------------------------
%	EDUCATION SECTION
%----------------------------------------------------------------------------------------

\begin{rSection}{Education}

{\bf The Ohio State University} \hfill {Columbus, OH} 
\\ Master's of Science (MS), Physics; Adviser: Christopher M. Hirata\hfill{December 2023}
\\ Doctorate of Philosophy (PhD), Physics; Adviser: Christopher M. Hirata\hfill{\textit{Expected 2026}}
\\ \small{\textit{Honors: Honorable Mention from NSF Graduate Research Fellowships Program,}}
\\ \hspace{10pt}\small{\textit{Dept. of Energy Office of Science Graduate Student Research (SCGSR) Fellowship Awardee}}

{\bf Whitman College} \hfill {Walla Walla, WA} 
\\ Bachelor of Arts in Physics \& Astronomy, Minor in Mathematics \hfill {May 2021}
\\ \small{\textit{Honors: Magna Cum Laude, Honors in Major, Walter A. Brattain Scholarship}}

\end{rSection}

%----------------------------------------------------------------------------------------
%	TECHNICAL STRENGTHS SECTION
%----------------------------------------------------------------------------------------


%	RELEVANT EXPERIENCE
%----------------------------------------------------------------------------------------

\begin{rSection}{Research Experience} 

\begin{rSubsection}{Kavli Institute for Particle Astrophysics and Cosmology, SLAC}{November 2024 - Present}{DOE SCGSR Fellow}{}
\item Analyzed observations from the LSST Commissioning Camera and LSST Camera to diagnose systematic errors in the point-spread function (PSF) fitting and modeling in the LSST Pipeline
\item Developing a physically motivated PSF model for the LSST Camera
\item Creating framework for data-sharing between LSST/Rubin and Roman telescopes in collaboration with both teams (ongoing)
\end{rSubsection}

\begin{rSubsection}{Dept. of Physics, the Ohio State University}{August 2022 - Present}{Graduate Research Assistant}{}
\item Analyzed noise data from \textit{Roman} detectors to understand the exact impact of noise on weak lensing measurement precision
\item Developing a Python program for removal of correlated noise from \textit{Roman} images in order to improve weak lensing, in alignment with the findings from the above work (ongoing)
\item Modified and expanded a C++ program testing different potential observing strategies for the Nancy Grace Roman Space Telescope (\textit{Roman}) mission, and constructed a sample observing plan for the High Latitude Survey for use in the \textit{OpenUniverse} simulations
\item Contributed ideas and communicated findings in regular meetings with advisor and collaborators
\item Collaborated on 4 papers describing the \textit{Roman} image processing simulations, their results, and the implications for \textit{Roman} science
\end{rSubsection}

% \begin{rSubsection}{Dept. of Physics, the Ohio State University}{August 2022 - May 2023}{Graduate Teaching Associate}{}
% \item Developed algorithms in Python to analyze \textit{Roman} detector read noise behavior and test its impact on measurements of weak gravitational lensing
% \item Wrote batch shell scripts for slurm workload manager on the Ohio State Supercomputer
 %
% \end{rSubsection}

\begin{rSubsection}{Exoplanet Exploration Program, Jet Propulsion Laboratory}{June 2020 - August 2020}{Intern}{}
\item Compiled sets of archival radial velocity (RV) data from 13 instruments for 54 target stars for future exoplanet direct imaging missions.
\item Analyzed RV data using multiple Python packages to find evidence of stellar/sub-stellar companions or activity cycles
\item Wrote, edited, and published a research manuscript on this work as primary author
\end{rSubsection}


\end{rSection}

%----------------------------------------------------------------------------------------
% LEADERSHIP + ACTIVITIES
%----------------------------------------------------------------------------------------
\begin{rSection}{Leadership \& Outreach} 

\begin{rSubsection}{Polaris Program, the Ohio State University}{April 2022 - Present}{Parliamentarian}{}
\item Planned and facilitated weekly meetings of the 12-person Polaris organization Leadership Team of grads and undergrads across different subfields of physics and astronomy
\item Created and documented new Recruitment \& Marketing, Grants, and High School Outreach committees
\item Presented a twice-yearly report to community stakeholders detailing the use of our \$60,000 budget, the successes of our outreach programs, and ideal areas of prospective growth
\end{rSubsection}


\begin{rSubsection}{\textit{URSA} Polaris Program, the Ohio State University}{}{Program Facilitator}{Summer 2024, Summer 2022}
\item Marketed URSA to over 50 incoming physics/astronomy students from underrepresented backgrounds via mail, email, phone, and social media
\item Organized logistics of the 15-student program, including housing, meals, transportation, technology access, classroom spaces, and communication channels between facilitators and participants
\item Designed and taught a 2-week engaging, group-work centered curriculum exploring the question \textit{``What is Time?'' }through the lens of astrophysics
\end{rSubsection}

\end{rSection}

%%%%%%%%%%%%%%%%%%%%%%%%%%%%%%%%%%%%%%%%%%%%%%%%%%%%
%% TEACHING EXPERIENCE
%%%%%%%%%%%%%%%%%%%%%%%%%%%%%%%%%%%%%%%%%%%%%%%%%%%%

\begin{rSection}{Teaching Experience}

\begin{rSubsection}{Dept. of Physics, the Ohio State University}{August 2022 - May 2023}{Graduate Teaching Associate}{}
\item Courses: ``Mechanics, Work and Energy, Thermal Physics" and ``E\&M, Waves, Optics, Modern Physics"
\item Delivered supplementary lectures, guided student group work sessions, and facilitated labs weekly
\item Developed engaging and educational problems and materials for weekly office hours and 3 exam review sessions; gathered and used student feedback to inform teaching strategies and focuses
\end{rSubsection}

\end{rSection}



%%%%%%%%%%%%%%%%%%%%%%%%%%%%%%%%%%%%%%%%%%%%%%%%%%%%
%% PUBLICATIONS
%%%%%%%%%%%%%%%%%%%%%%%%%%%%%%%%%%%%%%%%%%%%%%%%%%%%

\begin{rSection}{Scientific Publications}

\item \textbf{Laliotis, Katherine}, et al. “Analysis of Biasing from Noise from the Nancy Grace Roman Space Telescope: Implications for Weak Lensing.” Publications of the Astronomical Society of the Pacific, vol. 136, no. 12, 1 Dec. 2024, p. 124506, https://doi.org/10.1088/1538-3873/ad9bec. Accessed 19 Feb. 2025.

\item OpenUniverse, et al. “OpenUniverse2024: A Shared, Simulated View of the Sky for the next Generation of Cosmological Surveys.” ArXiv.org, 2024, arxiv.org/abs/2501.05632. Accessed 19 Feb. 2025.

\item Cao, Kaili, Hirata, Christopher M., \textbf{Laliotis, Katherine}, et al. “Simulating Image Coaddition with the Nancy Grace Roman Space Telescope: III. Software Improvements and New Linear Algebra Strategies.” ArXiv.org, 2024, arxiv.org/abs/2410.05442. Accessed 19 Feb. 2025.

\item Masaya Yamamoto, \textbf{Katherine Laliotis}, Emily Macbeth, Tianqing Zhang, Christopher M Hirata, M A Troxel, et al., Simulating image coaddition with the Nancy Grace Roman Space Telescope – II. Analysis of the simulated images and implications for weak lensing, Monthly Notices of the Royal Astronomical Society, Volume 528, Issue 4, March 2024, Pages 6680–6705, https://doi.org/10.1093/mnras/stae177

\item Christopher M Hirata, Masaya Yamamoto, \textbf{Katherine Laliotis}, Emily Macbeth, M A Troxel, Tianqing Zhang, Kaili Cao, et al., Simulating image coaddition with the Nancy Grace Roman Space Telescope – I. Simulation methodology and general results, Monthly Notices of the Royal Astronomical Society, Volume 528, Issue 2, February 2024, Pages 2533–2561, https://doi.org/10.1093/mnras/stae182

\item \textbf{Laliotis, Katherine}, Burt, J. A., Mamajek, E. E., Li, Z., Perdelwitz, V., et al. (2023). Doppler constraints on planetary companions to nearby sun-like stars: An archival radial velocity survey of southern targets for proposed NASA Direct Imaging Missions. The Astronomical Journal, 165(4), 176. https://doi.org/10.3847/1538-3881/acc067 

\end{rSection}

%%%%%%%%%%%%%%%%%%%%%%%%%%%%%%%%%%%%%%%%%%%%%%%%%%%%
%% PUBLIC WRITING
%%%%%%%%%%%%%%%%%%%%%%%%%%%%%%%%%%%%%%%%%%%%%%%%%%%%


\begin{rSection}{Public Writing}

\item \textbf{Laliotis, Katherine Helen.} “Europe’s Extremely Large Telescope Faces a New Dire Threat.” Scientific American, 30 Jan. 2025, www.scientificamerican.com/article/europes-extremely-large-telescope-faces-a-new-dire-threat/. Accessed 19 Feb. 2025.

\item \textbf{Laliotis, Katherine. }“Saving the Chandra X-Ray Observatory.” Undark Magazine, 19 Sept. 2024, undark.org/2024/09/19/opinion-saving-chandra-x-ray-observatory/.

\end{rSection}


%%%%%%%%%%%%%%%%%%%%%%%%%%%%%%%%%%%%%%%%%%%%%%%%%%%%
%% CONFERENCES/WORKSHOPS
%%%%%%%%%%%%%%%%%%%%%%%%%%%%%%%%%%%%%%%%%%%%%%%%%%%%

\end{document}
